\documentclass[11pt, a4paper]{article} 
% \usepackage[a4paper]{geometry}              
\usepackage{amsmath}
\usepackage{amssymb}
\usepackage{amsthm}
\usepackage{mathrsfs}
\usepackage{graphicx}
\usepackage{hyperref}
\hypersetup{
    colorlinks=true,  % hyperlinks will be coloured
    linkcolor=blue,   % hyperlink text will be blue
    citecolor=blue
}
\usepackage{xcolor}
% \usepackage{a4wide}

\newtheorem{theorem}{Theorem}[section]
\newtheorem{corollary}{Corollary}[section]
\newtheorem{lemma}[theorem]{Lemma}
\newtheorem{proposition}[theorem]{Proposition}
\newtheorem{definition}{Definition}
%\newenvironment{proof}[1][Proof]{\begin{trivlist}
%\item[\hskip \labelsep {\bfseries #1}]}{\end{trivlist}}

\newcommand{\R}{\mathbb{R}}
\newcommand{\C}{\mathbb{C}}
\newcommand{\E}{\mathbb{E}}
\newcommand{\fol}{\mathcal{F}}
\newcommand{\one}{\mathbb{1}}
\newcommand{\rank}{\text{rank}}
\newcommand{\detp}{{\det}_+}

\begin{document}
\title{Matrix completion and semidefinite matrics}

\author{Olaf Dreyer\thanks{OD Consulting}
\and Horst K\"ohler \thanks{Commerzbank AG}
\and Thomas Streuer \thanks{RIVACON GmbH} }

\date{\today}

\maketitle

\begin{abstract} Here comes the abstract.
\end{abstract}

\emph{AMS classification:} 15A15; 15A18; 15A83; 15B48
 \\
 
\emph{Keywords:} positive definite; positive semidefinite; matrix completions; 

\section{Introduction}
And some introduction.

The Schur complement needs to be introduced here.

\section{Partitioned matrices of maximal rank}\label{sec.partition}
Let the Hermitian matrix $H\in M_n(\C)$ be partitioned as follows:
\begin{equation}
	H = \begin{pmatrix}\label{eqn.defh}
		A & B \\
		B^\star & C
	\end{pmatrix},
\end{equation}
with $A\in M_k(\C)$, $C\in M_l(\C)$, $1\le k,l\le n-1$, $k+l=n$. Let $V = K\oplus L$ be the decomposition of $V\simeq \C^n$ in accordance with the partition of $H$ in equation (\ref{eqn.defh}) so that $A$ is a map from $K$ to $K$, and $C$ is a map from $L$ to $L$. Let us further assume that $H$ is positive semidefinite. Let $w\in N(A)\subset K$ be an element of the nullspace of $A$, i.e. let $A w = 0$. For $v=(w,0)^T\in V$ we then have 
\begin{equation}
	v^\star H v=0.
\end{equation}
Since $H$ is positive semidefinite this implies (see \cite{matrixanalysis}) that we already have
\begin{equation}
	Hv = \begin{pmatrix}
		0 \\
		B^\star w
	\end{pmatrix} = 0,
\end{equation}
or 
\begin{equation}\label{eqn.nulla}
	N(A) \subset N(B^\star).
\end{equation}
In a similar fashion we can establish that the null space of $C$ is contained in the null space of $B$:
\begin{equation}\label{eqn.nullc}
	N(C) \subset N(B).
\end{equation}
The relations for the null spaces are equivalent to these relations for the ranges of $A$ and $C$:
\begin{align}
	R(B) & \subset R(A) \\
	R(B^\star) & \subset R(C).
\end{align}
Because of these properties $H$ is said to have the \emph{column inclusion property}\cite{matrixanalysis}. It follows from equations (\ref{eqn.nulla}) and (\ref{eqn.nullc}) that 
\begin{equation}
	N(A) \oplus N(C) \subset N(H).
\end{equation}
This implies that the rank of $H$ is less than or equal to the sum of the ranks of $A$ and $C$:
\begin{align}
	\rank\ H & = n - \dim N(H) \\
	& \le n - (\dim N(A) + \dim N(C)) \\
	& = k-\dim N(A) + (n-k) - \dim N(C) \\
	& = \rank\ A + \rank\ C
\end{align}
We have equality if and only if $N(H) = N(A)\oplus N(C)$. In the following, matrices $H$ for which this equality holds, will be of particular interest to us which is why we make the following definition:

\begin{definition}
	Let $H$ be a positive semidefinite Hermitian matrix that is partitioned as in equation (\ref{eqn.defh}). We say that $H$ is of \emph{maximal rank} if and only if $N(H) = N(A)\oplus N(C)$. 
\end{definition}

When $H$ is of maximal rank, it vanishes on
\begin{equation}
	N(A)\oplus N(C),
\end{equation}
and is positive definite when restricted to the sum of the ranges of $A$ and $C$:
\begin{equation}
	R(A) \oplus R(C).
\end{equation} 
We will use this property to extend results that are valid for positive definite matrices to partitioned matrices of maximal rank. Before we can state these results we need to introduce one more notion.

\section{The generalized determinant}\label{sec.gendet}
A Hermitian matrix $H$ defines a non-singular map from its range $R(H)$ to its range $R(H)$. If $H$ is singular its determinant vanishes. The determinant thus contains no information about $H$ restricted to $R(H)$. To recover this information we introduce a generalized determinant $\detp$:

\begin{definition}\label{def.gendet}
	Let $H$ be Hermitian. Let $\bar H$ be $H$ restricted to the range of $H$:
	\begin{equation}
		\bar H = H \vert_{R(H)} : R(H) \longrightarrow R(H) \\
	\end{equation}
	For $H\ne 0$ we then set 
	\begin{equation}
		\detp H = \det \bar H.
	\end{equation}
	For $H=0$ we set $\detp H = 1$.
\end{definition}

We note a number of properties of the generalized determinant:

\begin{lemma}
	Let $H$ be Hermitian of rank $r\le n$. Let $\lambda_i, i=1, \ldots, n$, be the eigenvalues of $H$. Let us assume that they are ordered in such a way that $\lambda_i = 0$, for $i>r$. We then have:
	\begin{enumerate}
		\item If $H$ is of full rank (i.e. if $r=n$) we have
		\begin{equation}
			\detp H = \det H = \prod_i \lambda_i
		\end{equation}
		\item For $r<n$ we have
		\begin{equation}
			\detp H = \prod_{i=1}^r \lambda_i
		\end{equation}
		\item We have
		\begin{equation}
			\detp H = \lim_{\epsilon \rightarrow 0} \frac{\det( H + \epsilon I )}{\epsilon^{n-r}}
		\end{equation}
		\item For $c>0$ we have
		\begin{equation}
			\detp c H = c^{r} \detp H
		\end{equation}
	\end{enumerate}   
\end{lemma}

\begin{proof}
	All of these identities follow from the fact that $H = U D U^\star$ for a unitary $U$ and a diagonal $D$ that contains the eigenvalues of $H$ on the diagonal.
\end{proof}

We note that the generalized determinant does not have many of the properties the usual determinant has. In particular, it is not a continuous function of $H$, and the generalized determinant of the product of two matrices is not the product of the two generalized determinants. 

\section{The extensions}\label{sec.ext}
With the preparations in section \ref{sec.partition}, and the definition of the generalized determinant in the last section, we now want to extent three results: Fischer's inequality, the determinant equality for Schur's complement, and Banachiewicz's form of the inverse of a matrix.

\subsection{Fischer's inequality}
Fischer's inequality states that for a Hermitian positive semidefinite $H$ as in equation (\ref{eqn.defh}) we have (see e.g. \cite{horn}):
\begin{equation}
	\det H \le \det A\; \det C.
\end{equation}
Furthermore, if $H$ is positive definite, we have equality if and only if $B=0$. If $H$ is singular, the determinant of $H$ vanishes and the inequality is no longer much of a constraint. We can also no longer infer that $B=0$ in the case of equality. 

We can do better when $H$ is of maximal rank. In this case $H$ vanishes on $N(A)\oplus N(C)$ and is positive definite on 
\begin{equation}
	R(A)\oplus R(C).
\end{equation}
We now look at the restriction of $H$ and all its sub-matrices to this space and use the usual Fischer inequality there. As in definition \ref{def.gendet}, we denote the restriction of $H$ to its range by $\bar H$. We obtain
\begin{align}
	\detp H & = \det \bar H \\
	&\le \det \bar A\; \det \bar C \\
	& = \detp A\; \detp C.
\end{align}
Because we are looking at the restriction of $H$ to $R(A)\oplus R(C)$ and because it is positive definite on $R(A)\oplus R(C)$, we also get that $B$, when restricted to $R(C)$, vanishes if and only if equality holds above. Since $B$ vanishes on $N(C)$, this is the case if and only if
\begin{equation}
	B = 0.
\end{equation}
We thus have the following result:

\begin{proposition}\label{prop.fischer}
	Let a Hermitian $H$ be positive semidefinite and partitioned as in equation (\ref{eqn.defh}). If $H$ is of maximal rank then
	\begin{equation}
		\detp H \le \detp A\; \detp C,
	\end{equation} 
	with equality if and only if $B = 0$. 
\end{proposition}

Let us note that we arrived at this result in two steps. Because $H$ is positive semidefinite we know that $B$ restricted to $N(C)$ is zero. That the restriction of $B$ to the range $R(C)$ is also zero follows from Fischer's equality for the positive definite matrix that is $H$ restricted to $R(A)\oplus R(C)$.

\subsection{Schur complement}
We now turn our attention to the Schur complement. For a positive definite $H$ Schur showed that
\begin{equation}
	\det H = \det A\; \det H/A.
\end{equation}
We again focus our attention on $R(A)\oplus R(C)$ where $H$ is positive definite. Using the notation from the previous section we obtain:
\begin{align}
	\detp H & = \det \bar H \\
	& = \det \bar A \; \det \bar H / \bar A \\
	& = \detp A  \; \det \bar H / \bar A
\end{align}
We need to convince ourselves that the last determinant is equal to $\detp H/A$. To check this we need to show that 
\begin{align}
	N(H/A) & = N( C - B^\star A^+ B ) \\ 
	& = N(C).
\end{align}
We already know that $N(C) \subset N(H/A)$. To show equality let $l\in N(H/A)$ and set
\begin{equation}
	k = - A^+ B l.
\end{equation}
It then follows that 
\begin{equation}
	H\begin{pmatrix}
		k\\
		l
	\end{pmatrix} = 0.
\end{equation}
Since $H$ is of maximal rank this implies that $(k,l)^T\in N(A)\oplus N(C)$. In particular, we have $l\in N(C)$. We thus obtain our second result:

\begin{proposition}\label{prop.schur}
 Let a Hermitian $H$ be positive semidefinite and partitioned as in equation (\ref{eqn.defh}). If $H$ is of maximal rank then
	\begin{equation}
		\detp H = \detp A\; \detp H/A.
	\end{equation}	
\end{proposition}

\subsection{The inverse}
The result concerns the inverse of $H$. When $H$ is positive definite its inverse can be written as
\begin{equation}\label{eqn.bana}
	H^{-1} = \begin{pmatrix}
		A^{-1} + A^{-1}B(H/A)^{-1}B^\star A^{-1} & - A^{-1} B (H/A)^{-1} \\
		- (H/A)^{-1} B^\star A^{-1} & (H/A)^{-1}
	\end{pmatrix}.
\end{equation} 
This is a remarkable formula because to find the of $H$ we just need to invert $A$ and $H/A$. This formula was first established by Banachiewicz \cite{bana} (see also \cite[p.112]{frazer}). We now want to adapt this formula to our situation where $H$ might be singular but is of maximal rank. 

Again, we start by looking at the restriction $\bar H$ of $H$ to its range $R(A)\oplus R(C)$. $\bar H$ is positive definite and we can express its inverse in the form of equation (\ref{eqn.bana}). In the last section we established that the range and null spaces of $C$ and $H/A$ coincide so that by replacing the inverses with Moore-Penrose inverses we obtain the unique Moore-Penrose inverse of $H$.

\begin{proposition}\label{prop.bana}
	Let a Hermitian $H$ be positive semidefinite and partitioned as in equation (\ref{eqn.defh}). If $H$ is of maximal rank then its Moore-Penrose inverse is given by
	\begin{equation}\label{eqn.banapenrose}
	H^{+} = \begin{pmatrix}
		A^{+} + A^{+}B(H/A)^{+}B^\star A^{+} & - A^{+} B (H/A)^{+} \\
		- (H/A)^{+} B^\star A^{+} & (H/A)^{+}
	\end{pmatrix}.
	\end{equation}
\end{proposition}

This result can also be found in \cite{ouellette}. 

\section{Matrix completion}
Let $H$ be a Hermitian matrix in $M_n(\C)$ that is partitioned as follows:
\begin{equation}\label{eqn.defnewh}
	H = \begin{pmatrix}
		A & B & X \\
		B^\star & C & D \\
		X^\star & D^\star & E
	\end{pmatrix}
\end{equation}
To denote sub-matrices of $H$ we use the notation from \cite{matrixanalysis}. For a set $\alpha\subset \{1, \ldots, n\}$ we let
\begin{equation}
	H[\alpha]
\end{equation}
be the sub-matrix of $H$ with row and column indices in $\alpha$. For $\alpha,\beta\subset \{1, \ldots, n\}$ we let
\begin{equation}
	H[\alpha, \beta] 
\end{equation}
be the sub-matrix with row indices in $\alpha$ and column indices in $\beta$. Now let $\alpha$ and $\beta$ be such that
\begin{equation}
	H[\alpha] = \begin{pmatrix}
		A & B \\
		B^\star & C
	\end{pmatrix}
\end{equation}
and 
\begin{equation}
	H[\beta] = \begin{pmatrix}
		C & D \\
		D^\star & E
	\end{pmatrix}.
\end{equation}
For $\gamma = \alpha\cap\beta$ we then have
\begin{equation}
	H[\gamma] = C,
\end{equation}
and for the matrix $X$ in the upper right corner we have
\begin{equation}
	H[ \alpha-\gamma, \beta-\gamma ] = X.	
\end{equation}
We now want to know under what conditions we can choose $X$ so that the matrix $H$ is positive semidefinite. Before we state the result we note this helpful theorem:

\begin{theorem}\label{theo.albert}
	Let $H$ be Hermitian and partitioned as in equation (\ref{eqn.defh}). Then these two statements are equivalent:
	\begin{enumerate}
		\item $H$ is positive semidefinite.
		\item $A$ and $H/A$ are positive semidefinite, and $R(B)\subset R(A)$.
	\end{enumerate}	
\end{theorem}

A proof of this result can be found in \cite{albert}. We now have:

\begin{theorem}\label{theo.x}
	Let $H\in M_n(\C)$ be partitioned as in (\ref{eqn.defnewh}) and let $H[\alpha]$ and $H[\beta]$ be positive semidefinite. Setting
	\begin{align}
		X & = B C^+ D \\
		& = H[\alpha-\gamma, \gamma] H[\gamma]^+ H[\gamma, \beta-\gamma]
	\end{align}
	turns $H$ into a positive semidefinite matrix.
\end{theorem}

\begin{proof}
	We apply theorem \ref{theo.albert} to the positive semidefinite matrices $H[\alpha]$ and $H[\beta]$ to obtain:
	\begin{align}
		C & \ge 0 \\
		H[\alpha]/C & \ge 0 \label{eqn.hac}\\
		H[\beta]/C  & \ge 0 \label{eqn.hbc} \\
		R(B^\star) & \subset R(C) \label{eqn.bsc} \\				
		R(D) & \subset R(C) \label{eqn.dsc}
	\end{align}	
	The Schur complement $H/C$ is given by
	\begin{equation}
		H/C = \begin{pmatrix}
			A - B C^+ B^\star & X - B C^+ D \\
			X^\star - D^\star C^+ B^\star & E - D^\star C^+ D
		\end{pmatrix}.
	\end{equation}
	If we choose 
	\begin{equation}
		X = B C^+ D,
	\end{equation}
	and also recognize the expressions for $H[\alpha]/C$ and $H[\beta]/C$ we find
	\begin{equation}
		H/C = \begin{pmatrix}
			H[\alpha]/C & 0 \\
			0 & H[\beta]/C
		\end{pmatrix}.
	\end{equation}
	Because of equations (\ref{eqn.hac}) and (\ref{eqn.hbc}) we have
	\begin{equation}
		H/C \ge 0.
	\end{equation}
	Equations (\ref{eqn.bsc}) and (\ref{eqn.dsc}) then ensure that
	\begin{equation}
		R(B^\star\; D ) \subset R(C).
	\end{equation}
	Since we also have $C\ge 0$ we can use the theorem \ref{theo.albert} one more time, this time in the other direction, to obtain
	\begin{equation}
		H \ge 0.
	\end{equation} 
	This completes the proof.
\end{proof}

This result appeared already in \cite{smith}. We have provided a different proof that makes use of theorem \ref{theo.albert}. 

It turns out that the choice of $X$ in theorem \ref{theo.x} gives $H$ unique properties. It is in these characterizations of $X$ that we go beyond the results in \cite{smith} because we include the case in which $H$ is singular. The first result characterizes $X$ as the unique extension of $H$ that maximizes the determinant of $H$. If $H$ is non-singular we can just talk about the regular determinant of $H$. If $H$ is singular we have to use the generalized determinant that we introduced in section \ref{sec.gendet}.

\begin{theorem}
	Let $H\in M_n(\C)$ be partitioned as in (\ref{eqn.defnewh}) and let $H[\alpha]$ and $H[\beta]$ be positive semidefinite and of maximal rank. The choice 
	\begin{equation}
		X = B C^+ D
	\end{equation}
	is the unique choice for $X$ for which $H$ is positive semidefinite, of maximal rank and for which the (generalized) determinant is maximal.  
\end{theorem}

\begin{proof}
We have shown in the last theorem that $H$ is positive semidefinite if we set $X = B C^+ D$. $H$ is also of maximal rank. In general we have
\begin{equation}
	\rank\ H \le \rank\ A + \rank\ C + \rank\ E.
\end{equation}
For a positive semidefinite matrix rank is additive over the Schur complement (see \cite{ouellette}) so that we actually have equality:
\begin{align}
	\rank\ H & = \rank\ C + \rank\ H/C \\
	& = \rank\ C + \rank\ H[\alpha]/C + \rank\ H[\beta]/C \\
	& = \rank\ C + \rank\ A + \rank\ E
\end{align}
To establish the last equality we have used the assumption that both $H[\alpha]$ and $H[\beta]$ are of maximal rank. Thus $X=B C^+ D$ turns $H$ into a matrix of maximal rank. We now want to show that it is the only such choice that also maximizes the determinant.

Let us now assume that $H$ is positive semidefinite and of maximal rank so that we can make use of propositions \ref{prop.fischer} and \ref{prop.schur}. Because of proposition \ref{prop.schur} we have
\begin{equation}
	\detp H = \detp C \; \detp H/C.
\end{equation}
Because of proposition \ref{prop.fischer} we have
\begin{equation}
	\detp H/C \le \detp H[\alpha]/C + \detp H[\beta]/C,
\end{equation}
with equality if and only if
\begin{equation}
	X = B C^+ D.
\end{equation}
It follows that this $X$ is the unique choice that maximizes the determinant.
\end{proof}

For a non-singular matrix $H$ we can use the determinant to find the inverse of $H$ (see \cite{matrixanalysis}):
\begin{equation}\label{eqn.inverseformula}
	H^{-1} = \frac{1}{\det H}\left(\frac{\partial}{\partial h_{ij}}\det H \right)^T
\end{equation}
Because we are expressing $X$ in terms of the given sub-matrices $A, \ldots, E$ of $H$, it is clear that the derivative in equation (\ref{eqn.inverseformula}) vanishes for indices $i$ and $j$ that denote elements of $X$ itself. $H^{-1}$ has zeroes in those places where the matrix $X$ sits in $H$. It turns out that this characteristic uniquely determines $X$, even if $H$ is only positive semidefinite and we have to talk about the Moore-Penrose inverse of $H$ instead.

\begin{theorem}
	Let $H\in M_n(\C)$ be partitioned as in (\ref{eqn.defnewh}) and let $H[\alpha]$ and $H[\beta]$ be positive semidefinite and of maximal rank. The choice 
	\begin{equation}
		X = B C^+ D
	\end{equation}
	is the unique choice for $X$ for which $H$ is of maximal rank and for which $H^{+}$ has zeroes in those places where $X$ sits in $H$.
\end{theorem}

\begin{proof}
	Because $H$ is of maximal rank we can use proposition \ref{prop.bana} to express the Moore-Penrose inverse of $H$ in terms of $C^+$ and $(H/C)^+$. If $H^+$ has zeroes where $X$ is in $H$ then we must have
	\begin{equation}
		(H/C)^+ = \begin{pmatrix}
			(H[\alpha]/C)^+ & 0 \\
			0 & (H[\beta]/C)^+  
		\end{pmatrix},
	\end{equation}
	which can only be the case if $X=B C^+ D$.
\end{proof}

For completeness we give the Moore-Penrose inverse for $H$:

\begin{equation}
	H^+ = \begin{pmatrix}
		(H[\alpha]/C)^+ & -(H[\alpha]/C)^+ B C^+ & 0 \\
		- C^+ B^\star (H[\alpha]/C)^+ & \Xi & - C^+ D (H[\beta]/C)^+ \\
		0 & - (H[\beta]/C)^+ D^\star C^+ & (H[\beta]/C)^+ 
	\end{pmatrix},
\end{equation}
with
\begin{equation}
	\Xi = C^+ + C^+ B^\star (H[\alpha]/C)^+ B C^+ + C^+ D (H[\beta]/C)^+ D^\star C^+.
\end{equation}
\section{Conclusion}

\begin{thebibliography}{MM}
	\bibitem{matrixanalysis} Roger A. Horn, Charles R. Johnson, \emph{Matrix Analysis}, 2nd Edition, Cambridge University Press, 2013.
	\bibitem{schur} J. Schur, �ber Potenzreihen, die im Innern des Einheitskreises beschr�nkt sind, Journal f�r die reine und angewandte Mathematik \textbf{147}, 205 -- 232, 1917.
	\bibitem{bana} T. Banachiewicz, Zur Berechnung der Determinanten, wie auch der Inversen, und zur darauf basierten Aufl�sung der Systeme linearer Gleichungen, Acta Astronom. S\'er. C 3, 41--67, 1937.
	\bibitem{frazer} Frazer, R., Duncan, W., Collar, A. (1938). Elementary Matrices And Some Applications To Dynamics And Differential Equations, Cambridge University Press, 1938.
	\bibitem{ouellette} Ouellette, D. (1981). Schur complements and statistics Linear Algebra and its Applications  36, 187-295. https://dx.doi.org/10.1016/0024-3795(81)90232-9
	\bibitem{albert} Albert, A. (1969). Conditions for Positive and Nonnegative Definiteness in Terms of Pseudoinverses SIAM Journal on Applied Mathematics  17(2), 434-440. https://dx.doi.org/10.1137/0117041
	\bibitem{smith} Smith, R. (2008). The positive definite completion problem revisited Linear Algebra and its Applications  429(7), 1442-1452. https://dx.doi.org/10.1016/j.laa.2008.04.020	
	% older stuff
	\bibitem{hayns} Emilie Virginia Haynsworth, Determination of the inertia of a partitioned Hermitian matrix,  Linear  Algebra and its Applications \textbf{1}, 1968, 73--81.
	\bibitem{horn} Roger A. Horn, Basic  Properties  of  the  Schur  Complement. In Fuzhen Zhang (Ed.) The Schur complement and its application (17--46). Springer, 2005.
	\bibitem{rao} C. Radharkishna Rao, Linear Statistical Inference and its Applications, 2nd edition, John Wiley \& Sons, Inc, 2002.
	\bibitem{cottle} Richard W. Cottle, Manifestations of the Schur complement, Linear Algebra and its Applications \textbf{8}, 1974, 189--211. 
	\bibitem{roger} Penrose, R. (1955). A generalized inverse for matrices Mathematical Proceedings of the Cambridge Philosophical Society  51(3), 406-413. https://dx.doi.org/10.1017/s0305004100030401
\end{thebibliography}

\appendix

\section{Maybe ...}

\end{document}